\documentclass{letter}
%\documentclass{article}

\usepackage[utf8]{inputenc}
\usepackage{hyperref,cite}
\usepackage[a4paper, total={7in, 10in}]{geometry}
%\signature{Daniel Antunes, Jacques-Olivier Lachaud, Hugues Talbot}
%\address{}

\makeatletter
\newenvironment{thebibliography}[1]
     {\list{\@biblabel{\@arabic\c@enumiv}}%
           {\settowidth\labelwidth{\@biblabel{#1}}%
            \leftmargin\labelwidth
            \advance\leftmargin\labelsep
            \usecounter{enumiv}%
            \let\p@enumiv\@empty
            \renewcommand\theenumiv{\@arabic\c@enumiv}}%
      \sloppy
      \clubpenalty4000
      \@clubpenalty \clubpenalty
      \widowpenalty4000%
      \sfcode`\.\@m}
     {\def\@noitemerr
       {\@latex@warning{Empty `thebibliography' environment}}%
      \endlist}
\newcommand\newblock{\hskip .11em\@plus.33em\@minus.07em}
\makeatother

\signature{Daniel Antunes, Jacques-Olivier Lachaud, Hugues Talbot}
\begin{document}
\begin{letter}{Cover letter for paper ``An Elastica-driven Digital Curve Evolution Model for Image Segmentation''}
\opening{Dear editors,}

Our article is an extension version of the DGCI'19 paper ``Digital
Curvature Evolution Model for Image Segmentation''
\cite{antunes19}. We have included three new contributions. In section
three, we describe a new local combinatorial optimization method to
evolve a flow driven by the minimization of the digital Elastica
energy. The method is suitable for two multigrid convergent estimators
of curvature, and the results are illustrated with figures and a small
discussion about the running time. In section four, we extend the
digital evolution model described in \cite{antunes19}. In particular,
we generalize the curve evolution model and we define a family of
energies, each one giving rise to a new evolution model. We've
included a discussion with respect to the QPBOP optimization method
and how the number of unlabeled pixels varies in the family of
energies. Remarkably, QPBOP labels all variables for one of the
energies in the family. Finally, in section five, we compared our
segmentation method with two other algorithms: GrabCut
\cite{rother04grabcut} and Schoenemmans's \cite{schoenemann09linear},
and we illustrate the results with several figures.

Thank you for your time and consideration.

\closing{Yours Faithfully,}

\bibliographystyle{plain}
\bibliography{../jmiv.bib}

\end{letter}


%\end{letter}


\end{document}
